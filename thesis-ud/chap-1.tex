\chapter{Background}
Apart from general text content, structured information is also widely contained by digital document. Among these, a lot of mathematical content (including documents on Internet), are represented using markups like \LaTeX\  or MathML~\footnote{\url{http://www.w3.org/Math/}}, which is in a rich structural way. 
Information Retrieval on those structured data in mathematics language is not that well-studied or exhaustively covered by mainstream IR research, compared to that with general text. 
Thus it can be challenging yet very helpful given the contribution and importance of mathematics to our science. 

However, the structured sense of mathematical language, as well as many its semantic properties (see section~\ref{measure_sim}), makes general text retrieval models deficient to provide good search results. Through this paper, we have made our efforts to tackle some of these problems. 
Some of the ideas used in this paper deals with "tree structured" data in a general way, have the potential to be applied by other fields of structured data retrieval besides that from mathematical language. 

\section{Math IR Topics}

Mathematical information involves a wide spectrum of topics, 
we are, of cause, not focusing on every aspects in mathematical information retrieval. 
It is good to clarify our concentration in this paper here, by first listing a set of concentrations that a mathematical information retrieval topics may be classified into,
and define our target field of study.

\pagebreak
Listed here, are considered four possible concentrations of topic for mathematical information retrieval:

\begin{enumerate}
\item Boolean or Similarity Search
\item Math Detection and Recognition
\item Evaluation, Derivation and Calculation
\item Other topics 
\end{enumerate}

The first one is doing mathematical information retrieval by searching, and finding the most relevant context of document that matches the query, very similar to the most common ways that other general text search engines will do, by boolean or similarity search~\cite{iir}. 
The only difference is, the query may contain mathematical expressions. 
Instance of such search engine can be useful in many ways, for example, student may utilize it to know which identity can be applied to a formulae in order to give a proof of that formulae.
This is the area where we focus in this paper. Specifically, we are proposing a series of methods for similarity search of math content. And our method is using query only in \LaTeX\ markup 
(some math-aware search engines~\footnote{WolframAlpha: \url{https://www.wolframalpha.com/} and Zentralblatt math \\from MathWebSearch: \url{http://search.mathweb.org/zbl/}} support queries in mathematical formulae and normal text together), and return documents ordered by score which indicates the similarity degree. 

Digital mathematical content document can also be in an image format (e.g. generated by a handwritten query), thus to retrieve these information involves detection or recognition. Inspired by the advances from deep learning, we may foresee a large potential to be explored on topics related to this. 

Because the nature of mathematical language, a query (e.g. an algebra expression) can be evaluated and potentially derived into an alternate form, or calculated. 
The result value of evaluation or derived form may also be considered being relevant to that query. 
These potentially require a system to handle symbolic or value calculation, or even a good knowledge of derivation rules implied by different mathematical expression
(e.g. computational engine \textit{Symbolab}~\footnote{Symbolab Web Search: \url{http://www.symbolab.com}}).

Besides the first three concentrations, there are many other topics. Knowledge mining, for example, will need deeper level of understanding on math content. A typical goal of this topic is to give a solution or answer based on information retrieved from math content. e.g. ``Find an article related to the \textit{Four Color Theorem}"~\cite{ntcirtopic}.

These concentrations somehow overlap in some cases, for example, some derivation can be used to better assess the similarity between math formulae, e.g. $\dfrac{a + b}{c}$ and $\dfrac{a}{c} + \dfrac{b}{c}$ should be considered as relevant.
Therefore even boolean or similarity search possibly involves certain level of understanding of mathematics. In terms of similarity, however, we only address the measurement for structural and symbol differences in this paper, without considering further topics lured from measuring math content similarity, such as evaluation or derivation.

\section{Issues in Measuring Similarity}
\label{measure_sim}
Unlike general text content, mathematical language, by its nature, has many differences from other textual documents, there are a number of new problems in measuring mathematical expression similarity. 
Among these, we select and focus on those regarding to structural similarity and symbolic differences between expressions. 
At the same time trying to respect the semantical information inferred from structure or symbols in mathematical expressions.
But even without caring about the possible derivations and high level knowledge inference, there are still many new problems.

Firstly, differences of symbols, structure and possible semantic rules in mathematics should be captured, and not one by one, but in an cooperative manner to measure similarity. 
To illustrate this point, we know that only respecting symbolic information is of course not sufficient in mathematical language. e.g. $ax+(b+c)$ in most cases is not equivalent to $(a+b)x+c$ (although they have the same set of symbols).
And the order of tokens in math expression can be commutative in some cases but not always. For example, commutative property in math makes $a+b=b+c$ for addition operation, but on the other hand $\dfrac a b$ is most likely not equivalent to $\dfrac b a$.
These make many general text search methods (e.g. \textit{bag of words} model, \textit{tf-idf} weighting) inadequate. 
Moreover, symbols can be used interchangeably to represent the same meaning, e.g. $a^2+b^2=c^2$ and $x^2+y^2=z^2$. 
However, interchangeability comes with some constrains to maintain the same semantical meaning, that is, changes of symbols in expression preserve more syntactic similarity when changes are made by substitution. e.g. For query $x(1+x)$, expression $a(1+a)$ are considered more relevant than $a(1+b)$. 

Secondly, how we evaluate structural similarity between expressions is a question. A complete query may structurally be a part of a document, or only some parts of a query match somewhere in a document expression.
In cases when a set of matches occur within some measure of ``distance", we may consider them to contribute similarity as a whole, but when matches occur ``far away" for a query expression, then under the semantic implication of mathematics, they probably will not contribute the similarity degree in any way.
We need a metres to score these similarity under certain criteria and set up standard and rules for relevance assessments.

Lastly, trying to capture semantic information from expressions will help measure similarity but introduce ambiguity. 
Apart from the cases covered in \cite{parsing_tex}, semantic incorrect written markups, which is somehow common in many online documents, e.g. writing ``sin" in \LaTeX\ markup instead of macro ``\textbackslash sin", will make it difficult to tell whether it is a product of three symbols or a \textit{sine} function, thus need to disambiguate. 
And depending on what level of semantical meaning we want to capture, ambiguity cases can be different. 
Consider $f(2x+1)$, if we want to know if $f$ is a function rather than a variable, the only possibility is looking for implicit contexts, but we can nevertheless always think of it as a product without losing the possibility to search similar expression like $f(1 + 2y)$, the same way goes reciprocal $a^{-1}$ and inverse function $f^{-1}$. 
Most often, even if no semantic ambiguity occurs, efforts are needed to capture some semantical meanings. e.g. In $\displaystyle\int f(x) \dfrac{\mathrm{d}x}{\sin x}$ and $\sin 2 \pi$, it is not easy to figure out, without a little knowledge on integral or trigonometric function, that integral is applied to $\dfrac {f(x)} {\sin x}$ and the scope applied by sine function is $2 \pi$. 


\section{Related Work}
Mathematical boolean or similarity search is not a new topic, and conference in this topic is getting increasingly research attention and the proposed systems have progressed considerably~\cite{ov}. 

Originally text-based systems are using existing models to deal with mathematical search. 
DLMF project from NIST ~\cite{Youssef0}~\cite{Youssef1} developed an TexSN (i.e. Textualization, Serialization/Scoping, and Normalization) process to convert math to text. 
The MIaS system~\cite{mias} try to reorder commutative operations and normalize variable and constance into unified symbols. 
These texted-based approaches build a math search engine on top of conventional text search technology. 
The conversion process loses considerable structural, and captures little semantics.

Our system Cowpie \footnote{demo page: \url{infolab.ece.udel.edu:8912/cowpie/}} \cite{WolframAlpha}

MathML vs LaTeX

distributed indexing to quickly search massive

Further more, a query may be specified with wildcards and thus will match any document with an expression substitution to that wildcard. 
