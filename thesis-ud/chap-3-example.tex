\subsection{Illustrated by an Example}
\label{secIllu}

We illustrate the method introduced in this chapter by a simple example here. 
Given a query expression $ax(a+b)$ and document expression $ax + (b+a)by$, here we show how we search the relevant document using this query and how the relevance score between them is calculated by mark-and-cross algorithm.

\begin{figure}
\begin{minipage}[b]{2.65in}
\begin{center}
\resizebox{2.4in}{!}{
%%%%%%%%%%%%%%%%%%%%%%%%%%%%%%%%%%%%%%%%%%%%%%%%
% \begin{tikzpicture}[anchor=mid,>=latex',line join=bevel,]
\begin{tikzpicture}[>=latex',line join=bevel,]
  \pgfsetlinewidth{1bp}
\Huge%
\begin{scope}
  \pgfsetstrokecolor{black}
  \definecolor{strokecol}{rgb}{1.0,1.0,1.0};
  \pgfsetstrokecolor{strokecol}
  \definecolor{fillcol}{rgb}{1.0,1.0,1.0};
  \pgfsetfillcolor{fillcol}
  \filldraw (0bp,0bp) -- (0bp,180bp) -- (234bp,180bp) -- (234bp,0bp) -- cycle;
\end{scope}
\begin{scope}
  \pgfsetstrokecolor{black}
  \definecolor{strokecol}{rgb}{1.0,1.0,1.0};
  \pgfsetstrokecolor{strokecol}
  \definecolor{fillcol}{rgb}{1.0,1.0,1.0};
  \pgfsetfillcolor{fillcol}
  \filldraw (0bp,0bp) -- (0bp,180bp) -- (234bp,180bp) -- (234bp,0bp) -- cycle;
\end{scope}
  \pgfsetcolor{black}
  % Edge: ad5 -- b4
  \draw [] (179.35bp,72.765bp) .. controls (185.17bp,61.456bp) and (192.89bp,46.437bp)  .. (198.7bp,35.147bp);
  % Edge: ti6 -- a1
  \draw [] (84.43bp,146.83bp) .. controls (72.02bp,134.77bp) and (54.269bp,117.51bp)  .. (41.796bp,105.38bp);
  % Edge: ad5 -- a3
  \draw [] (162.65bp,72.765bp) .. controls (156.83bp,61.456bp) and (149.11bp,46.437bp)  .. (143.3bp,35.147bp);
  % Edge: ti6 -- ad5
  \draw [] (113.57bp,146.83bp) .. controls (125.98bp,134.77bp) and (143.73bp,117.51bp)  .. (156.2bp,105.38bp);
  % Edge: ti6 -- x2
  \draw [] (99bp,143.7bp) .. controls (99bp,132.85bp) and (99bp,118.92bp)  .. (99bp,108.1bp);
  % Node: b4
\begin{scope}
  \definecolor{strokecol}{rgb}{0.0,0.0,0.0};
  \pgfsetstrokecolor{strokecol}
  \draw (207bp,18bp) ellipse (27bp and 18bp);
  \draw (207bp,18bp) node {$^4_v b$};
\end{scope}
  % Node: ad5
\begin{scope}
  \definecolor{strokecol}{rgb}{0.0,0.0,0.0};
  \pgfsetstrokecolor{strokecol}
  \draw (171bp,90bp) ellipse (27bp and 18bp);
  \draw (171bp,90bp) node {$^5_a +$};
\end{scope}
  % Node: ti6
\begin{scope}
  \definecolor{strokecol}{rgb}{0.0,0.0,0.0};
  \pgfsetstrokecolor{strokecol}
  \draw (99bp,162bp) ellipse (27bp and 18bp);
  \draw (99bp,162bp) node {$^6_t \times$};
\end{scope}
  % Node: a1
\begin{scope}
  \definecolor{strokecol}{rgb}{0.0,0.0,0.0};
  \pgfsetstrokecolor{strokecol}
  \draw (27bp,90bp) ellipse (27bp and 18bp);
  \draw (27bp,90bp) node {$^1_v a$};
\end{scope}
  % Node: a3
\begin{scope}
  \definecolor{strokecol}{rgb}{0.0,0.0,0.0};
  \pgfsetstrokecolor{strokecol}
  \draw (135bp,18bp) ellipse (27bp and 18bp);
  \draw (135bp,18bp) node {$^3_v a$};
\end{scope}
  % Node: x2
\begin{scope}
  \definecolor{strokecol}{rgb}{0.0,0.0,0.0};
  \pgfsetstrokecolor{strokecol}
  \draw (99bp,90bp) ellipse (27bp and 18bp);
  \draw (99bp,90bp) node {$^2_v x$};
\end{scope}
%
\end{tikzpicture}
%%%%%%%%%%%%%%%%%%%%%%%%%%%%%%%%%%%%%%%%%%%%%%%%
} 
\\[5mm]
query expression $ax(a+b)$
\end{center}
\end{minipage}
\hspace*{.18in}
\begin{minipage}[b]{2.65in}
\begin{center}
\resizebox{3.3in}{!}{
%%%%%%%%%%%%%%%%%%%%%%%%%%%%%%%%%%%%%%%%%%%%%%%%
\begin{tikzpicture}[>=latex',line join=bevel,]
  \pgfsetlinewidth{1bp}
\Huge%
\begin{scope}
  \pgfsetstrokecolor{black}
  \definecolor{strokecol}{rgb}{1.0,1.0,1.0};
  \pgfsetstrokecolor{strokecol}
  \definecolor{fillcol}{rgb}{1.0,1.0,1.0};
  \pgfsetfillcolor{fillcol}
  \filldraw (0bp,0bp) -- (0bp,252bp) -- (342bp,252bp) -- (342bp,0bp) -- cycle;
\end{scope}
\begin{scope}
  \pgfsetstrokecolor{black}
  \definecolor{strokecol}{rgb}{1.0,1.0,1.0};
  \pgfsetstrokecolor{strokecol}
  \definecolor{fillcol}{rgb}{1.0,1.0,1.0};
  \pgfsetfillcolor{fillcol}
  \filldraw (0bp,0bp) -- (0bp,252bp) -- (342bp,252bp) -- (342bp,0bp) -- cycle;
\end{scope}
  \pgfsetcolor{black}
  % Edge: ad0 -- ti8
  \draw [] (192.43bp,218.83bp) .. controls (180.02bp,206.77bp) and (162.27bp,189.51bp)  .. (149.8bp,177.38bp);
  % Edge: ti9 -- a1
  \draw [] (243bp,143.7bp) .. controls (243bp,132.85bp) and (243bp,118.92bp)  .. (243bp,108.1bp);
  % Edge: ad7 -- b4
  \draw [] (179.35bp,72.765bp) .. controls (185.17bp,61.456bp) and (192.89bp,46.437bp)  .. (198.7bp,35.147bp);
  % Edge: ti8 -- b5
  \draw [] (116.19bp,148.81bp) .. controls (96.365bp,135.96bp) and (65.338bp,115.85bp)  .. (45.597bp,103.05bp);
  % Edge: ti9 -- x2
  \draw [] (257.57bp,146.83bp) .. controls (269.98bp,134.77bp) and (287.73bp,117.51bp)  .. (300.2bp,105.38bp);
  % Edge: ad7 -- a3
  \draw [] (162.65bp,72.765bp) .. controls (156.83bp,61.456bp) and (149.11bp,46.437bp)  .. (143.3bp,35.147bp);
  % Edge: ti8 -- y6
  \draw [] (126.65bp,144.76bp) .. controls (120.83bp,133.46bp) and (113.11bp,118.44bp)  .. (107.3bp,107.15bp);
  % Edge: ti8 -- ad7
  \draw [] (143.35bp,144.76bp) .. controls (149.17bp,133.46bp) and (156.89bp,118.44bp)  .. (162.7bp,107.15bp);
  % Edge: ad0 -- ti9
  \draw [] (215.35bp,216.76bp) .. controls (221.17bp,205.46bp) and (228.89bp,190.44bp)  .. (234.7bp,179.15bp);
  % Node: b4
\begin{scope}
  \definecolor{strokecol}{rgb}{0.0,0.0,0.0};
  \pgfsetstrokecolor{strokecol}
  \draw (207bp,18bp) ellipse (27bp and 18bp);
  \draw (207bp,18bp) node {$^4_v b$};
\end{scope}
  % Node: ti8
\begin{scope}
  \definecolor{strokecol}{rgb}{0.0,0.0,0.0};
  \pgfsetstrokecolor{strokecol}
  \draw (135bp,162bp) ellipse (27bp and 18bp);
  \draw (135bp,162bp) node {$^8_t \times$};
\end{scope}
  % Node: ti9
\begin{scope}
  \definecolor{strokecol}{rgb}{0.0,0.0,0.0};
  \pgfsetstrokecolor{strokecol}
  \draw (243bp,162bp) ellipse (27bp and 18bp);
  \draw (243bp,162bp) node {$^9_t \times$};
\end{scope}
  % Node: ad0
\begin{scope}
  \definecolor{strokecol}{rgb}{0.0,0.0,0.0};
  \pgfsetstrokecolor{strokecol}
  \draw (207bp,234bp) ellipse (27bp and 18bp);
  \draw (207bp,234bp) node {$^{10}_a +$};
\end{scope}
  % Node: ad7
\begin{scope}
  \definecolor{strokecol}{rgb}{0.0,0.0,0.0};
  \pgfsetstrokecolor{strokecol}
  \draw (171bp,90bp) ellipse (27bp and 18bp);
  \draw (171bp,90bp) node {$^7_a +$};
\end{scope}
  % Node: a1
\begin{scope}
  \definecolor{strokecol}{rgb}{0.0,0.0,0.0};
  \pgfsetstrokecolor{strokecol}
  \draw (243bp,90bp) ellipse (27bp and 18bp);
  \draw (243bp,90bp) node {$^1_v a$};
\end{scope}
  % Node: b5
\begin{scope}
  \definecolor{strokecol}{rgb}{0.0,0.0,0.0};
  \pgfsetstrokecolor{strokecol}
  \draw (27bp,90bp) ellipse (27bp and 18bp);
  \draw (27bp,90bp) node {$^5_v b$};
\end{scope}
  % Node: a3
\begin{scope}
  \definecolor{strokecol}{rgb}{0.0,0.0,0.0};
  \pgfsetstrokecolor{strokecol}
  \draw (135bp,18bp) ellipse (27bp and 18bp);
  \draw (135bp,18bp) node {$^3_v a$};
\end{scope}
  % Node: x2
\begin{scope}
  \definecolor{strokecol}{rgb}{0.0,0.0,0.0};
  \pgfsetstrokecolor{strokecol}
  \draw (315bp,90bp) ellipse (27bp and 18bp);
  \draw (315bp,90bp) node {$^2_v x$};
\end{scope}
  % Node: y6
\begin{scope}
  \definecolor{strokecol}{rgb}{0.0,0.0,0.0};
  \pgfsetstrokecolor{strokecol}
  \draw (99bp,90bp) ellipse (27bp and 18bp);
  \draw (99bp,90bp) node {$^6_v y$};
\end{scope}
%
\end{tikzpicture}
%%%%%%%%%%%%%%%%%%%%%%%%%%%%%%%%%%%%%%%%%%%%%%%%
} 
\\[5mm]
document expression $ax + (b+a)by$
\end{center}
\end{minipage}
\caption{Example query/document expression representation}\label{expGraph}
\end{figure}

The query expression and document expression are represented by operation trees $T_q$ and $T_d$ in figure~\ref{expGraph}. 
Instead of only denoting the operation symbols at the internal nodes and the variable/constant symbols at the leaf nodes, we use the notation $^i_l S$ or $^i S$ to denote a node with symbol $S$ labeled by $l$ (i.e. $\ell(S)=l$) with vertex number $i$. 
The three different possible labels used here are $v,\ t$ and $a$, standing for unified name ``variable", ``times" and ``add" respectively.
To be concise and descriptive, we will interchangeably use either $^i S$ notation or $q_i$ (or $p_i$) to represent the leaf-root path in a query (or document) operation tree where the leaf vertex $i$ resides.

Firstly, the generated path sets for $T_q$ and $T_d$ are:
$$
\begin{aligned}
g(T_q) &= \{ 1\cdot6,\; 2\cdot6,\; 3\cdot5\cdot6,\; 4\cdot5\cdot6\} \\
g(T_d) &= \{ 5\cdot8\cdot10,\; 6\cdot8\cdot10,\; 3\cdot7\cdot8\cdot10,\; 4\cdot7\cdot8\cdot10,\; 1\cdot9\cdot10,\; 2\cdot9\cdot10\}
\end{aligned}
$$
And the labeled path sets for each of the two are:
$$
\begin{aligned}
\ell\left(g(T_q)\right) &= \{ v\cdot t,\; v\cdot a\cdot t \} \\
\ell\left(g(T_d)\right) &= \{ v\cdot t\cdot a,\; v\cdot a\cdot t \cdot a \}
\end{aligned}
$$
Because $\ell\left(g(T_q)\right) \cdot a = \ell\left(g(T_d)\right)$, which follows 
$$
\ell(g(T_q)) \cdot \hat{a} \subseteq \ell(g(T_d))
$$
where $\hat{a} = a$, 
we know $T_d$ is searchable by $T_q$, so there exists a path $\hat{a}$ we can search to append after every labeled path in set $\ell\left(g(T_d)\right)$ so that we will find $T_d$ by intersecting all the formula trees indexed (in index $\Pi$) in these paths.

Secondly, by the implications from observation~\#1 of section~\ref{observationlabel}, we get the candidate set for each of the path in $T_q$:
$$
\begin{aligned}
C_{q_1} &= \{ p_5,\; p_6 \} \\
C_{q_2} &= \{ p_5,\; p_6 \} \\
C_{q_3} &= \{ p_3,\; p_4 \} \\
C_{q_4} &= \{ p_3,\; p_4 \} 
\end{aligned}
$$

In addition, get the list $L$ containing all the query paths in $T_q$ sorted by symbol and its occurrence in all path symbols, 
$$
\mathrm{QList} = q_1,\ q_3,\ q_4,\ q_2.
$$

\begin{table}
\begin{center}
\renewcommand{\arraystretch}{2}
\begin{tabular}{|c"D{.}{}{2.5}|D{.}{}{2.5}|D{.}{}{2.5}|D{.}{}{2.5}|}
\hline
\textbf{current score} & \multicolumn{4}{c|}{$0.9$}\\ \thickhline
\textbf{bound variable} & 
\multicolumn{1}{c|}{$B_a = 1$} & 
\multicolumn{2}{c|}{$B_b = 1.8$ (max)} & 
\multicolumn{1}{c|}{$B_y = 0.9$} \\ \thickhline
\backslashbox{\textbf{query path}}{\textbf{document path}} & 
\multicolumn{1}{c|}{$^3 a$} & 
\multicolumn{1}{c|}{$^4 b$} & 
\multicolumn{1}{c|}{$^5 b$} & 
\multicolumn{1}{c|}{$^6 y$} \\ \thickhline
$^1 a$ & & & \multicolumn{1}{c|}{$0.9$} & \multicolumn{1}{c|}{$0.9$}\\ \hline
$^3 a$ & \multicolumn{1}{c|}{$1$} & \multicolumn{1}{c|}{$0.9$} & & \\ \hline
\end{tabular}
\renewcommand{\arraystretch}{1}
\end{center}
\caption{First two iterations of example score evaluation}\label{figure_it2}
\end{table}

Lastly, we can calculate the symbolic similarity degree between the two expressions by going through each query path from $\mathrm{QList}$ in order and apply mark-and-cross algorithm (use $\alpha = 0.9$).
Then the first two iterations which calculates the matching score for bound variable $a$ in $T_q$ can be illustrated by table~\ref{figure_it2}.

\begin{table}
\begin{center}
\renewcommand{\arraystretch}{2}
\begin{tabular}{|c"D{.}{}{2.5}|D{.}{}{2.5}|D{.}{}{2.5}|D{.}{}{2.5}|}
\hline
\textbf{current score} & \multicolumn{4}{c|}{$0.9 + 0.9 = 1.8 $}\\ \thickhline
\textbf{bound variable} & 
\multicolumn{1}{c|}{$B_a = 0.9$ (max)} & 
\multicolumn{2}{c|}{$B_b = 0$ } & 
\multicolumn{1}{c|}{$B_y = 0$} \\ \thickhline
\backslashbox{\textbf{query path}}{\textbf{document path}} & 
\multicolumn{1}{c|}{$^3 a$} & 
\multicolumn{1}{c|}{$^4 b$} & 
\multicolumn{1}{c|}{$^5 b$} & 
\multicolumn{1}{c|}{$^6 y$} \\ \thickhline
$^1 a$ & & & \multicolumn{1}{c|}{$0.9$} & \multicolumn{1}{c|}{$0.9$}\\ \hline
$^3 a$ & \multicolumn{1}{c|}{$1$} & \multicolumn{1}{c|}{$0.9$} & & \\ \hhline{="="="="="}
$^4 b$ & \multicolumn{1}{c|}{$0.9$} & \notableentry & \notableentry & \\ \hline
\end{tabular}
\renewcommand{\arraystretch}{1}
\end{center}
\caption{3rd iteration of example score evaluation}\label{figure_it3}
\end{table}

\begin{table}
\begin{center}
\renewcommand{\arraystretch}{2}
\begin{tabular}{|c"D{.}{}{2.5}|D{.}{}{2.5}|D{.}{}{2.5}|D{.}{}{2.5}|}
\hline
\textbf{current score} & \multicolumn{4}{c|}{$1.8 + 0.9 = 2.7$}\\ \thickhline
\textbf{bound variable} & 
\multicolumn{1}{c|}{$B_a = 0$} & 
\multicolumn{2}{c|}{$B_b = 0$} & 
\multicolumn{1}{c|}{$B_y = 0.9$ (max)} \\ \thickhline
\backslashbox{\textbf{query path}}{\textbf{document path}} & 
\multicolumn{1}{c|}{$^3 a$} & 
\multicolumn{1}{c|}{$^4 b$} & 
\multicolumn{1}{c|}{$^5 b$} & 
\multicolumn{1}{c|}{$^6 y$} \\ \thickhline
$^1 a$ & & & \multicolumn{1}{c|}{$0.9$} & \multicolumn{1}{c|}{$0.9$}\\ \hline
$^3 a$ & \multicolumn{1}{c|}{$1$} & \multicolumn{1}{c|}{$0.9$} & & \\ \hhline{="="="="="}
$^4 b$ & \multicolumn{1}{c|}{$0.9$} & \notableentry & \notableentry & \\ \hhline{="="="="="}
$^2 x$ & \notableentry & \notableentry & \notableentry & \multicolumn{1}{c|}{$0.9$} \\ \hline
\end{tabular}
\renewcommand{\arraystretch}{1}
\end{center}
\caption{4th iteration of example score evaluation}\label{figure_it4}
\end{table}

Each path of query bound variable $a$ is compared with that from document path in its candidate set, 
and each resulting $\mathrm{sim}(a,a')$ value is accumulated on the corresponding document bound variable $B_v$.
Then the current score is also calculated, by accumulating the max $B_v$ value among all the bound variable $v$ in document expression.
The tags associated with document path $^4b$ and $^5b$ are marked as crossed state, to prevent path $^4b$ and $^5b$ from being compared in future iterations.
For the next query bond variable $b$ in $\mathrm{QList}$, 
first zero every $B_v$ value, 
then calculate its matching score with the document paths in its candidate sets in a similar manner 
except we have skipped some document paths in candidate sets because they are crossed in the previous iteration.
Table~\ref{figure_it3} shows the intermediate results before iterate to the next query bond variable in $\mathrm{QList}$.
Finally, we use the same way to process the last query bond variable $x$. 
Its query path $^2x$ is matched with document path $^6 y$ with $\mathrm{sim}(^2 x,\ ^6 y)$ value equals to $0.9$. 
Then the only non-zero $B_v$ value from document bound variable $y$ will contribute to final symbolic similarity score which in the end is added up to $2.7$ (see table~\ref{figure_it4}). 


Now we have finished our symbolic similarity evaluation between given query expression and document expression. 
We can also infer that the matching-depth $d$ is $|\hat{a}| = 1$, matching factor 
$f(d) = \dfrac{1}{1 + d} = 0.5$, 
and the matching-ratio $r=\dfrac{|g(T_q)|}{|g(T_d)|} = \dfrac{4}{6} \approx 0.67$. 
The final score tuple $(2.7,\ 0.5,\ 0.67)$ is used to determine the similarity degree in general and rank the document expression.
